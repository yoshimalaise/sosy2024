\section{Introduction}
The use of Intelligent Tutoring Systems~(ITS) to adapt the recommendation of exercises based on individual learners has proven to be effective~\cite{kulik2016effectiveness}. Evidence suggests that more personalised learning leads to increased learner agency, self-reliance and motivation~\cite{prain2013personalised}. However, the move towards smart learning systems is not without costs. Intelligent Tutoring Systems typically require that all of the exercises are stored in one centralised system that is often also monitored by teachers and persons who are responsible for ultimately grading the students. Previous studies revealed that implementing electronic performance monitoring can result in lower job satisfaction, increased stress levels, reduced autonomy, and it can further be perceived as a violation of trust~\cite{siegel2022impact}. In ideal circumstances, the learning environments should also offer a safe space for students to practise on problems they find interesting or challenging, without having to worry about how they will be perceived by teachers. To give an example, a student might be worried that the perception of the teacher might change if they are still shown to be struggling with material from previous years.

% Unlike other related work focusing on the decentralisation and data ownership of learning progress~\cite{chowdhury2020towards,mikroyannidis2020blockchain}, we do not aim for a solution that ensures the validity of a user's progress, but rather focus on enhancing the exercises that can be generated and recommended for users based on their performance. We aim for a solution where (a) \emph{the same user progress can be used across multiple learning applications} to provide a smooth flow state and (b) \emph{a student can opt in to share progress with individual applications for training the model}, but without having to give permissions to teachers on all exercises they have performed. Instead, students can freely decide which exercise results they want to share with their educators. 

In the following, we propose a solution to this problem based on the Solid specification where (a) \emph{the same user progress can be used across multiple learning applications} to provide a smooth flow state and (b) \emph{a student can opt in to share progress with individual applications for training the model}, but without having to provide teachers access to their entire performance for individual exercises. We discuss the solution by answering some open questions posed by Malaise and Signer~\cite{Malaise2023Explorotron}. They proposed the \emph{Explorotron} prototype for programming education. In this prototype, there are multiple smaller sub-applications called \emph{study lenses}, each taking a source code file and generating exercises based on certain difficulty levels matching the \mbox{PRIMM}~methodology~\cite{sentance2019teaching}. While their tool offers the possibility to suggest learning experiences based on logical progression through built-in study lenses, the authors conclude with a few major challenges to overcome:

\begin{itemize}
    \item Students can generate exercises from any source file, so generated content is never used by other students, leading to a cold start problem that makes recommendations challenging.
    \item Students should be free to decide which data they want to share and with whom they want to share it.
    \item Third-party developers should be able to contribute custom exercise generators (study lenses) transparently---the same recommendations/profiling should work across the board. 
\end{itemize}