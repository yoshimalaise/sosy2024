\section{Conclusions}
In this paper, we analysed a use case of a digital learning platform that helps students to learn programming but currently has some open challenges on how personalisation and interoperability with third-party developers might be achieved without taking away the learner's ownership of their data. We then proposed a novel solution to support those tasks by allowing users to store all their exercise results in Solid Pods and giving them full control on how and when applications can access this data. In this work, we assumed that both the users and the application developers could be fully trusted. If more validation and authority is needed, additional layers can be built on top of this system, for example by using a blockchain to verify claims, as described in~\cite{chowdhury2020towards, mikroyannidis2020blockchain}.
